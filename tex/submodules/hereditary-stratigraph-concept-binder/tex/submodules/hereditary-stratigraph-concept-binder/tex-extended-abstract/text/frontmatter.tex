\title{ Hereditary Stratigraphy: Genome Annotations to Enable Phylogenetic Inference over Distributed Populations }

%%
%% The "author" command and its associated commands are used to define
%% the authors and their affiliations.
%% Of note is the shared affiliation of the first two authors, and the
%% "authornote" and "authornotemark" commands
%% used to denote shared contribution to the research.
\author{Matthew Andres Moreno}
\orcid{0000-0003-4726-4479}
\affiliation{%
  \institution{Michigan Statue University}
  \city{East Lansing}
  \state{Michigan}
  \country{USA}
  \postcode{48103}
}

\author{Emily Dolson}
% \email{dolsonem@msu.edu}
\orcid{0000-0001-8616-4898}
\affiliation{%
  \institution{Michigan Statue University}
  \city{East Lansing}
  \state{Michigan}
  \country{USA}
  \postcode{48103}
}

\author{Charles Ofria}
% \email{ofria@msu.edu}
\orcid{0000-0003-2924-1732}
\affiliation{%
  \institution{Michigan Statue University}
  \city{East Lansing}
  \state{Michigan}
  \country{USA}
  \postcode{48103}
}


%%
%% By default, the full list of authors will be used in the page
%% headers. Often, this list is too long, and will overlap
%% other information printed in the page headers. This command allows
%% the author to define a more concise list
%% of authors' names for this purpose.
\renewcommand{\shortauthors}{Moreno et al.}

\begin{CCSXML}
<ccs2012>
   <concept>
       <concept_id>10010147.10010341.10010366.10010369</concept_id>
       <concept_desc>Computing methodologies~Simulation tools</concept_desc>
       <concept_significance>500</concept_significance>
       </concept>
   <concept>
       <concept_id>10010147.10010257.10010293.10011809.10011812</concept_id>
       <concept_desc>Computing methodologies~Genetic algorithms</concept_desc>
       <concept_significance>500</concept_significance>
       </concept>
 </ccs2012>
\end{CCSXML}

\ccsdesc[500]{Computing methodologies~Simulation tools}
\ccsdesc[500]{Computing methodologies~Genetic algorithms}

\keywords{phylogenetics, decentralized algorithms, digital evolution, genetic algorithms, genetic programming}

\begin{abstract}
Phylogenies provide direct accounts of the evolutionary trajectories behind evolved artifacts in genetic algorithm and artificial life systems.
Phylogenetic analyses can also enable insight into evolutionary and ecological dynamics such as selection pressure and frequency dependent selection.
Traditionally, digital evolution systems have recorded data for phylogenetic analyses through perfect tracking where each birth event is recorded in a centralized data structures.
This approach, however, does not easily scale to distributed computing environments where evolutionary individuals may migrate between a large number of disjoint processing elements.
To provide for phylogenetic analyses in these environments, we propose an approach to enable phylogenies to be inferred via heritable genetic annotations rather than directly tracked.
We introduce a ``hereditary stratigraphy'' algorithm that enables efficient, accurate phylogenetic reconstruction with tunable, explicit trade-offs between annotation memory footprint and reconstruction accuracy.
In particular, we demonstrate an approach that enables estimation of the most recent common ancestor between two individuals with fixed relative accuracy irrespective of lineage depth while only requiring logarithmic annotation space complexity with respect to lineage depth.
This approach can estimate, for example, MRCA generation of two genomes within 10\% relative error with 95\% confidence up to a depth of a trillion generations with genome annotations smaller than a kilobyte.
We also simulate inference over known lineages, recovering up to 85.70\% of the information contained in the original tree using a 64-bit annotation.
\end{abstract}


\maketitle
