% ../../tex/lib/importpath.tex
\makeatletter
  \def\importpathx{\import@path}
\makeatother

\begin{singlespacing}
\csvreader[
  longtable=cccccc,
  table head=\caption{
    Comparison of phylogenetic reconstruction quality between stratum retention policies.
    Reconstruction quality measured as clustering information distance (lower is better), mutual clustering information (higher is better), and generalized Robinson-Foulds similarity (higher is better) between reconstructed phylogeny and ground truth phylogeny \citep{smith2020information, smith2020treedist}.  }\label{tab:reconstruction-quality-results-by-policy} \\
    \toprule \thead{Tree Comparison Metric} & \thead{Selection\\ Scheme} & \thead{Num\\ Differentia\\ Bits} & \thead{Target\\ Num\\ Column\\ Bits} & \thead{Recency- \\Proportional \\ Resolution\\ Score} & \thead{Tapered\\ Depth-\\ Proportional\\ Resolution\\ Score} \\ \midrule\endfirsthead
    \caption*{\tablename{} \thetable{} (cont'd)} \\
    \toprule \thead{Tree Comparison Metric} & \thead{Selection\\ Scheme} & \thead{Num\\ Differentia\\ Bits} & \thead{Target\\ Num\\ Column\\ Bits} & \thead{Recency- \\Proportional \\ Resolution\\ Score} & \thead{Tapered\\ Depth-\\ Proportional\\ Resolution\\ Score} \\ \midrule\endhead
    \bottomrule\endfoot,
  late after line=\\,
]{\importpathx submodules/hereditary-stratigraph-concept-binder/binder/reconstruction-quality/outplots/reconstruction_quality_results_by_policy.csv}{}
{\csvcoli & \csvcolii & \csvcoliii & \csvcoliv & \csvcolv & \csvcolvi}
\end{singlespacing}
